\documentclass{sig-alternate-05-2015}

\makeatletter
\def\@copyrightspace{\relax}
\makeatother

\usepackage{verbatim}
\usepackage{graphicx}
\usepackage{gensymb}
\usepackage{float}
\usepackage[normalem]{ulem}
\useunder{\uline}{\ul}{}
\usepackage{url}

\pagenumbering{arabic}

\begin{document}

\title{2to3D a User-Friendly Parametric CAD Program\\for Laser Cutting}
\subtitle{CS701 Final Project Report \\ 12/19/17}

\numberofauthors{2} % 2 authors in this file

\author{
% You can add or remove blocks as needed.
%
% 1st. author
\alignauthor
Leo McElroy\\
       \affaddr{Middlebury College}\\
       \affaddr{Middlebury, VT}\\
       \email{lmcelroy@middlebury.edu}
% 2nd. author
\alignauthor
Ben Brown\\
       \affaddr{Middlebury College}\\
       \affaddr{Middlebury, VT}\\
       \email{bwbrown@middlebury.edu}
}

\maketitle

\begin{abstract}

We created a user-friendly CAD (Computer-Aided Design) program for laser cutting which integrates the desired features of vector drawing software and 3D parametric CAD programs. We implemented basic drawing, editting, and transformation tools. The program is capable of outputting SVGs (the Scalable Vector Graphics image format). Geometric constraints are solved using Cassowary.js - an implementation of the Simplex Algorithm. Linear constraints (coincidence, fixing of points, vertical, and horizontal) are efficient and robust. Nonlinear constraints (distance, line angles, parallel, and perpendicular) are not naturally accommodate by the Simplex Algorithm and required clever workarounds to function, consequently they are less robust. The program is accessible by virtue of being open-source and a web application which can run in any modern web-browser. We were successful in creating an accessible and functional parametric design program which simplifies the laser cutting workflow, however the project requires improvement before we foresee adaptation by laser cutting hobbyists. Future development of the project would include building out drawing features, adding circle and arc shape primitives, raster image import, utilizing a geometric constraint solver that accommodates nonlinear equations (such as gradient descent), and improving state handling of the program to support undo/redo tools.

\end{abstract}


\keywords{CS701; \LaTeX;}

\section{Introduction}

In the the last ten years the popularization of 3D printers and makerspaces has caused interest in digital fabricators to blossom. Digital fabricators are machines which use computer controlled tools to produce 3D objects. The digital fabrication workflow begins in a computer-aided design (CAD) program where the user produces electronic schematics. These schematics are then brought into a computer-aided manufacturing (CAM) program where the user creates tool paths for a computer numerical control (CNC) machine, this process outputs instructions for the fabrication machine. Finally these instructions are uploaded to a digital fabricator and the schematic is recreated in the medium worked by the fabrication machine. Depending on the machine this medium can be plastic, paper, cardboard, metal, stone, wood, glass, or even organic material. The digital fabrication workflow is depicted in Figure \ref{fig:digiFabWorkflow}.

\begin{figure}[H]
  \includegraphics[width=\linewidth]{digiFabWorkflow.jpg}
  \caption{The digital fabrication workflow from CAD to CAM to digital fabricator.}
  \label{fig:digiFabWorkflow}
\end{figure}

Personal digital fabricators, or fabricators intended for use outside of an industrial manufacturing setting (think personal computer versus mainframe), include 3D printers, CNC mills, and laser cutters. Laser cutters are often considered the best balance of accessibility and usefulness of these digital fabricators. A laser cutter works by directing a high powered laser through a focusing head which is attached to a gantry. The gantry controls the position of the head and therefore the location of the laser on a workpiece that rests on a bed underneath the gantry. A top-view schematic of a laser cutter is depicted in Figure~\ref{fig:lasersystem}.

\begin{figure}[H]
  \includegraphics[width=\linewidth]{lasersystem.jpg}
  \caption{Top-view schematic of a laser cutter adapted from Ref. \cite{laos}.}
  \label{fig:lasersystem}
\end{figure}

Laser cutters are capable of cutting and engraving. Most commercial laser cutters will cut plastic, paper, cardboard, and wood up to a quarter inch thick. They can also engrave all these materials and most types of metal and glass.

There are two main paradigms for modeling in CAD programs - direct modeling and parametric modeling. In direct modeling the user interacts directly with the geometry. This is typically through transformations such as dragging, rotating, or scaling. In parametric modeling the design program utilizes a geometric constraint solver that allows the user to specify constraints and dimensions on geometry. The design is then automatically adjusted to satisfy these constraints when the user modifies geometry directly. Figure~\ref{fig:parametricProgram} depicts the common constraint capabilities of a parametric design program.

\begin{figure}[H]
  \includegraphics[width=\linewidth]{parametricProgram.jpg}
  \caption{Screenshot of parametric CAD program (Onshape) which shows common geometric constraints.}
  \label{fig:parametricProgram}
\end{figure}

Laser cutters are typically 2-dimensional fabricators which means there is no $z$-axis adjustability during fabrication. Consequently, prevalent drawing programs such as Adobe Illustrator and CorelDRAW were adopted for creating designs intended for laser cutting. Vector graphics, a scale-agnostic format where images are represented mathematically, are typically used for generating cutting paths. Rasters, a format where images are represented by pixels, are typically interpreted as engravings when laser cutting.

The issue that motivated our project is that the laser cutting workflow is often made unnecessarily cumbersome by users switching between various design programs. These include parametric CAD programs, with features intended for designing 3D real objects, and high-powered drawing programs, with features intended for creating intricate drawings. Our project aims to unify the essential features for laser cutting found in each of these design softwares into one program.

In Section 2 we present the current issue with the laser cutting workflow and design process. In Section 3 we briefly review programs with some of the features we were interested in incorporating into 2to3D. In Section 4 we describe the basic shape tools and how we handle geometric constraints using Cassowary.js. In Section 5 we cover the rest of the functionality and capabilities of the 2to3D program. And in Section 6 we discuss the success of the project, review the major design principal of 2to3D, and describe extensions to our work which will result in 2to3D becoming a viable substitute for existing programs used for laser cutting. 

\section{Problem Statement}

Formally describe the problem you are solving/studying. Provide the details of
the problem, and any background that is necessary for a reader to understand
the problem, methods, and results. This is where you may want to articulate any
assumptions your research makes, and provide any important definitions. This is
also a good section to provide any explanatory figures that illustrate your
problem. This section should be approximately 1-2 pages in length.


\section{Related Work}

%Provide a review of prior work that is similar to yours. Who else has
%solved/researched this problem or similar problems? What was their approach?
%How is your research new and different? Here you can describe shortcomings in
%previous work that your work complements and or improves upon. This section
%should be approximately 0.5-1 page in length depending on the amount of related
%work.

\begin{itemize}
  \item \textbf{Easel from Inventables}: Easel is a web-based CAD program for CNC (computer numerical control) milling. It is non-parametric and integrated with CAM (computer aided manufacturing), this means it is capable of outputting G-Code.
  \item \textbf{Vectr}: Vectr is a simple vector drawing program capable of running in the browser. The drawing features are very similar to what we plan to implement.
  \item \textbf{Onshape}: Onshape is a commercial 3D parametric CAD program which is also web-based and integrated with CAM.
  \item \textbf{SolveSpace}:
  \item \textbf{jSketcher}: 
\end{itemize}


\section{Methods}

This section should include the following:

\begin{itemize}
		
\item	A description of the methods you are using (which itself could have
	sub-sections). If you have created a new application, you should describe
	which tools you have used to do so and the steps of your implementation. You
	should also briefly describe each of the tools you used (e.g. Ruby on Rails,
	MongoDB, D3, etc.). If you have proposed a new algorithm to solve a problem,
	you should provide the details of your algorithm along with pseudocode. If
	your algorithm is quite complex, you should describe its running time.  

\item Description of any data you used and explanation of why you chose this
	particular data (possibly organized into sub-sections).

\item Link to a GitHub, BitBucket or other repository if relevant.

\end{itemize}

\section{Results}

\subsection*{Other Capabilities}

We produced a highly accessible CAD program suitable for laser cutting. The accessibility of 2to3D can be attributed to the fact that the program runs entirely in the browser (thus requiring no additional software beyond a modern browser), and that the program strips away all non-essential drawing features and all 3D modeling features. Currently 2to3D only supports the creation of drawings intended for translation into cutting paths. Figure~\ref{fig:usingProgram} depicts a full digital fabrication workflow using 2to3D. The figure depicts the design of a press-fit box. The laser cutter available for use utilized CorelDRAW for CAM so the drawing was exported from 2to3D as a SVG and imported to CorelDRAW for printing.

\begin{figure}[H]
  \includegraphics[width=\linewidth]{usingProgram.jpg}
  \caption{Full digital fabrication workflow using 2to3D. The laser cutter we had access to used CorelDraw for CAM so the drawing was created in 2to3D and imported to CorelDraw just for printing.}
  \label{fig:usingProgram}
\end{figure}

2to3D has a simple design that displays all tools to the user without being overwhelming. This design inspired by ``lean-operating practices" also immediately and constantly conveys program state information by highlighting the current tool and changing the cursor style. Figures \ref{fig:screenshot} and \ref{fig:screenshot2} are screenshots of the program.

\begin{figure}[H]
  \includegraphics[width=\linewidth]{screenshot.png}
  \caption{Screenshot of 2to3D program depicting example bezier and rectangle.}
  \label{fig:screenshot}
\end{figure}

\begin{figure}[H]
  \includegraphics[width=\linewidth]{screenshot2.png}
  \caption{Screenshot of 2to3D program depicting rotation tool in action. Note the angle on display next to the highlights tool in the right toolbox.}
  \label{fig:screenshot2}
\end{figure}

2to3D supports three shape primitives (line, bezier and freehand) which can be used to create more complex shapes. The program supports both linear and non-linear constraints but linear constraints are far more robust. The program also supports direct transformations - rotations, moves, and scales - that break constraints of transformed objects. This allows the user to make direct edits to drawing geometry that will exactly reflect user's intention. In addition to these three aspects which were detailed in Section 4, 2to3D also supports a myriad of other features. The drawing area allows zooming in/out and panning and the workpiece size is adjustable which facilitates creation of designs on any scale. SVG exports draw dimensions from workpiece size which simplifies the CAM process if the user decides to match drawing dimensions to laser cutter bed size. Hotkeys support swift and smooth workflows and copy and paste capabilities allow users to quickly create redundant geometry. Geometry can also be deleted to correct errors. A save and upload feature also allows users to continue work on unfinished projects or to share designs with others. A full list of features can be found in Table~\ref{tab:capabilities}.

\begin{table}[H]
\centering
\caption{Capabilities of 2to3D.}
\label{tab:capabilities}
\begin{tabular}{|l|l|l|l|}
\hline
\multicolumn{2}{|l|}{{\ul \textbf{Drawing}}}                                                                                                                                                        & \multicolumn{2}{l|}{{\ul \textbf{Navigation}}}                                                                                                                                               \\
\multicolumn{2}{|l|}{\begin{tabular}[c]{@{}l@{}}Freehand\\ Rectangle\\ Polyline\\ Bezier\end{tabular}}                                                                                              & \multicolumn{2}{l|}{\begin{tabular}[c]{@{}l@{}}Pan\\ Zoom in/out\end{tabular}}                                                                                                               \\ \hline
\multicolumn{2}{|l|}{{\ul \textbf{Direct Transformations}}}                                                                                                                                         & \multicolumn{2}{l|}{{\ul \textbf{Display Options}}}                                                                                                                                          \\
\multicolumn{2}{|l|}{\begin{tabular}[c]{@{}l@{}}Move\\ Rotate\\ Scale\end{tabular}}                                                                                                                 & \multicolumn{2}{l|}{\begin{tabular}[c]{@{}l@{}}Lengths\\ Transformation values\end{tabular}}                                                                                                 \\ \hline
\multicolumn{2}{|l|}{{\ul \textbf{Constraints}}}                                                                                                                                                    & \multicolumn{2}{l|}{{\ul \textbf{Other}}}                                                                                                                                                    \\
\multicolumn{2}{|l|}{\begin{tabular}[c]{@{}l@{}}Horizontal (line)\\ Vertical (line)\\ Coincident (points)\\ Fixed (points)\\ Parallel (lines)\\ Perpendicular (lines)\\ Length (line)\end{tabular}} & \multicolumn{2}{l|}{\begin{tabular}[c]{@{}l@{}}Click and drag points\\ Download SVG\\ Save drawing\\ Upload drawing\\ Copy/Paste\\ Delete\\ Hotkeys\\ Workpiece size adjustable\end{tabular}} \\ \hline
\end{tabular}
\end{table}

\subsection*{Link to Video Demo and Program}
A demonstration of the program's features can be found at: \newline \underline{\url{https://www.youtube.com/watch?v=QDdJkAFzwLU}}. \newline

The program itself can be accessed while on Middlebury College's network at: \newline \underline{\url{basin.cs.middlebury.edu:3002}}.

\section{Discussion}

This section should begin with a brief summary of your results. Next, provide a
more detailed synopsis of your results: What new knowledge do they offer? What
lessons did you learn? What is the main take-away message of your work?
Finally, you should provide a brief critique of your own work: Point out the
specific attributes that you feel are extremely positive and note any
weaknesses or limitations. Discuss how your project could be improved. You can
also discuss possible extensions of your work. This section should be
approximately 1 page in length.

%ACKNOWLEDGMENTS are optional
\section{Acknowledgments}

This section is optional; it is a location for you to acknowledge grants,
assistance, etc. For example, this report template was adapted from Prof.
Christman.


\bibliography{references} 
\bibliographystyle{IEEEtran}

\end{document}
