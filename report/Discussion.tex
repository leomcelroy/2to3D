\section{Discussion}

%This section should begin with a brief summary of your results. Next, provide a
%more detailed synopsis of your results: What new knowledge do they offer? What
%lessons did you learn? What is the main take-away message of your work?
%Finally, you should provide a brief critique of your own work: Point out the
%specific attributes that you feel are extremely positive and note any
%weaknesses or limitations. Discuss how your project could be improved. You can
%also discuss possible extensions of your work. This section should be
%approximately 1 page in length.

Highly accessible but not the most robust parametric program. Users interested in using open-source web based programs for CAM can easily use 2to3D in conjunction with fab modules.

-specification of transformation values by user when direct transforming

We accomplished our main goal of creating a parametric web-based vector drawing program, but as with all software engineering, the work is never done. Future work would include the following goals:

- Build out shape primitives to include circles and arcs, as well as associated constraints (such as tangent).

- Improve the program?s state-handling to enhance user experience. This could involve introducing atomic operations which would allow tracking of a file?s timeline.

- Create a JavaScript library that can efficiently solve systems of nonlinear equations using gradient descent.

- Integrate 2to3D with existing open-source web-based CAM software such as MIT?s Center for Bits and Atoms? ?fab modules.?


